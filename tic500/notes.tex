\documentclass{report}

\usepackage{amsmath}

\begin{document}

\section{Important features of tilted window delay stages}
\begin{itemize}
    \item polarized transmission; rotation axes should orient normal to beam polarization
    \item small, but tunable dynamic range (a fraction of the travel length of the window)
    \item no beam path folding (can be retrofit inline quickly)
    \item non-linear step size.  Step value can be customized with e.g. microstepping or via gearhead systems
    \item easy DIY construction, low motor loads, no linear actuation (which is more expensive)
    \item color-dependent delay values
\end{itemize}

\section{Design notes}
\begin{itemize}
    \item the optimal angle range is around the Brewster angle, where transmission is high.  obvious limits of oblique angle due to clipping
    \item can determine angle/beam relationship by orienting normal and measuring the change path length (which gives n*L)
    \item thickness of optic glass is limited to by group velocity dispersion, which will distort the pulse; see https://www.edmundoptics.com/knowledge-center/application-notes/lasers/ultrafast-dispersion/
\end{itemize}

\begin{equation}\label{eq:pulse_broadening}
    \tau_\text{out} = \tau_\text{in} \sqrt{1 + \frac{4\ln{2} \times GDD}{\tau_\text{in}^2}^2}
\end{equation}
where $GDD=GVD \times \ell$ and $GVD=\frac{\partial}{\partial\omega} \frac{1}{v_g} = \frac{\partial^2 k}{\partial\omega^2}$.  GVD typically has units of $\text{fs}^2 / \text{mm}$.
A GDD of $\sim 10^5 \ \text{fs}^2$ can distort picosecond pulses.

\begin{equation}\label{eq:v_g}
    v_g = \frac{c}{n(\omega) + \omega \frac{\partial n}{\partial\omega}}
\end{equation}

\section{Math for the optical paths}


\end{document}